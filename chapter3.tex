\documentclass{article}%
\usepackage{amsmath}%
\usepackage{amsfonts}%
\usepackage{amssymb}%
\usepackage{graphicx}
\usepackage{enumitem}
%-------------------------------------------
\newtheorem{theorem}{Theorem}
\newtheorem{acknowledgement}[theorem]{Acknowledgement}
\newtheorem{algorithm}[theorem]{Algorithm}
\newtheorem{axiom}[theorem]{Axiom}
\newtheorem{case}[theorem]{Case}
\newtheorem{claim}[theorem]{Claim}
\newtheorem{conclusion}[theorem]{Conclusion}
\newtheorem{condition}[theorem]{Condition}
\newtheorem{conjecture}[theorem]{Conjecture}
\newtheorem{corollary}[theorem]{Corollary}
\newtheorem{criterion}[theorem]{Criterion}
\newtheorem{definition}[theorem]{Definition}
\newtheorem{example}[theorem]{Example}
\newtheorem{exercise}[theorem]{Exercise}
\newtheorem{lemma}[theorem]{Lemma}
\newtheorem{notation}[theorem]{Notation}
\newtheorem{problem}[theorem]{Problem}
\newtheorem{proposition}[theorem]{Proposition}
\newtheorem{remark}[theorem]{Remark}
\newtheorem{solution}[theorem]{Solution}
\newtheorem{summary}[theorem]{Summary}
\newcommand\abs[1]{\left|#1\right|}
\newcommand\I{\textbf{i}}
\newenvironment{proof}[1][]{\begin{samepage}\textbf{Proof #1} }{\rule{0.5em}{0.5em} \end{samepage}}
\setlength{\textwidth}{7.0in}
\setlength{\oddsidemargin}{-0.35in}
\setlength{\topmargin}{-0.5in}
\setlength{\textheight}{9.0in}
\setlength{\parindent}{0.3in}
\begin{document}

\begin{flushright}
\textbf{Brandon Toner \\
\today}
\end{flushright}

\begin{center}
\textbf{MATH 438: Introduction to Complex Variables \\
Proofs} \\
\end{center}

\begin{enumerate}
    \setcounter{enumi}{2}
    \item Chapter 3
    \begin{enumerate}[label*=\arabic*.]
        \setcounter{enumii}{2}
        \item %3.3
        \begin{proof}\\
            Assume $f$ is differentiable at $z$. Thus $\lim_{h\to 0} \frac{f(z+h)-f(z)}{h}$ does not approach $\infty$, and therefore $f(z+h)-f(z)$ must approach $0$. Which implies $f$ is continous at $z$.
        \end{proof}
        \setcounter{enumii}{9}
        \item %3.10
        \begin{enumerate}[label=(\alph*)]
            \item %3.10.a
            \begin{eqnarray*}
                \frac{d z^2}{dz} &=& \frac{(z+dz)^2 -z^2}{dz} \\
                                 &=& \frac{z^2+2zdz+dz^2-z^2}{dz} = \frac{2zdz+dz^2}{dz} \\
                                 &=& 2z+dz \\
                                 &=& 2z
            \end{eqnarray*}
            \item %3.10.b
            \begin{eqnarray*}
                \frac{d z^3}{dz} &=& \frac{(z+dz)^3 -z^3}{dz} \\
                                 &=& \frac{z^3+3z^2dz+3dz^2z+dz^3-z^3}{dz} = \frac{3z^2dz+3dz^2z+dz^3}{dz} \\
                                 &=& 3z^2+3zdz+dz^2\\
                                 &=& 3z^2
            \end{eqnarray*}
        \end{enumerate}
        \setcounter{enumii}{16}
        \item %3.16
            Assume 3.15
            \begin{eqnarray*}
                f &=& u+\I v \\
                f_x &=& u_x + \I v_x \\
                f_y &=& u_y + \I v_y \\
                f_x &=& f_y /\I \text{ by 3.15} \\
                u_x + \I v_x &=& (u_y + \I v_y)/\I = v_y -\I u_y \\
                &\implies& u_x = v_y \text{ and }  u_y =-v_x
            \end{eqnarray*}
        \setcounter{enumii}{19}
        \item %3.20
            \begin{eqnarray*}
                f &=&= u+0\I\\
                f_x &=& u_x \\
                f_y &=& u_y \\
                u_x = f_x &=& -\I f_y = -\I u_y \text{ by Complex Cauchy-Riemann} \\
                &\implies& u_x = u_y = 0 \text{ since $u_x, u_y \in \mathbb{R}$}
            \end{eqnarray*}
        \item %3.21
            \begin{enumerate}[label=(\alph*)]
                \item %3.21.a
                    \begin{eqnarray*}
                        u_r &=& \cos(\theta) \\
                        v_r &=& \sin(\theta) \\
                        u_\theta &=& -r \sin(\theta) \\
                        v_\theta &=& r \cos(\theta) \\
                        r u_r &=& r \cos(\theta) = v_\theta  \\
                        r v_r &=& r \sin(\theta) = -u_\theta
                    \end{eqnarray*}
            \end{enumerate}
        \setcounter{enumii}{24}
        \item %3.25
            Let $f$ be a $C^1$ analytic function.
            \begin{eqnarray*}
                \frac{\partial f}{\partial \bar{z}} &=& 0.5 (f_x - f_y/i) \\
                                                    &=& 0.5 (f_x - f_x) = 0 \text{ by Cauchy-Riemann}
            \end{eqnarray*}
        \setcounter{enumii}{30}
        \item %3.31
        \begin{eqnarray*}
            e^z &=& e^{x+y\I} \\
            (e^z)_x &=& e^{x+y\I} \\
            (e^z)_y &=& \I e^{x+y\I} \\
            (e^z)_x &=& (e^z)_y / \I \\
                   &\implies& e^z \text{ is analytic}
        \end{eqnarray*}
        \setcounter{enumii}{33}
        \item %3.34
            \begin{eqnarray*}
                \abs{e^z} &=& \abs{e^x(\cos(y)+\I\sin(y))} \\
                          &=& \abs{e^x}\abs{\cos(y)+\I\sin(y)} \\
                          &=& \abs{e^x}
            \end{eqnarray*}
        \item %3.35
        \begin{enumerate}[label=(\alph*)]
            \item %3.35.a
                All of $\mathbb{C}$ except for $0$
            \item %3.35.b
            \begin{eqnarray*}
                e^{z+2\pi\I} &=& e^{x+(y+2\pi)\I} \\
                             &=& e^x(\cos(y+2\pi)+\I\sin(y+2\pi)) \\
                             &=& e^x(\cos(y)+\I\sin(y)) \\
                             &=& e^z
            \end{eqnarray*}
            \item %3.35.c
            \item %3.35.d
                Since $e^z$ is periodic of period $2\pi\I$ there exists $n\in\mathbb{Z}$ such that $0 \leq y+2n\pi\I < 2\pi$ and $e^{z + 2n\pi\I} = e^z$
        \end{enumerate}
        \item %3.36
        Let $z=t(x+y\I)=xt+yt\I$
        \begin{eqnarray*}
            e^z &=& e^{xt+yt\I} \\
                &=& e^{xt}(\cos(yt)+ \I \sin(yt))
        \end{eqnarray*}
        $e^z$ is a spiral in the complex plane, with radius $e^{xt}$ and angle $yt$
        \item %3.37
        \begin{proof}[$\overline{e^z}=e^{\bar{z}}$]
            \begin{eqnarray*}
                \overline{e^z} &=& \overline{e^x(\cos(y) + \I \sin(y))} \\
                               &=& e^x(\cos(y)-\I\sin(y)) \\
                               &=& e^x(\cos(-y)+\I\sin(-y)) \\
                               &=& e^{\bar{z}}
            \end{eqnarray*}
        \end{proof}
        \item %3.38
        \begin{eqnarray*}
            e^z &=& e^x(\cos(y)+\I\sin(y)) \\
            e^w &=& e^u(\cos(v)+\I\sin(v)) \\
        \end{eqnarray*}
        \item %3.39
        \begin{proof}[$\sin(z)$ is entire]
        \begin{eqnarray*}
            \sin(z) &=& (e^{\I z}-e^{-\I z})/(2 \I) \\
            \sin_x(z) &=& (\I e^{\I z}+\I e^{-\I z}) / (2\I) = (e^{\I z}+e^{-\I z})/(2) \\
            \sin_y(z) &=& (-e^{\I z} - e^{-\I z})/(2\I) \\
            1/\I &=& -\I \\
            \sin_x(z) &=& \sin_y(z)/\I
        \end{eqnarray*}
        \end{proof}\\
        \begin{proof}[$\cos(z)$ is entire]
        \begin{eqnarray*}
            \cos(z) &=& (e^{\I z}+e^{-\I z})/2 \\
            \cos_x(z) &=& (\I e^{\I z}-\I e^{-\I z}) / 2 \\
            \cos_y(z) &=& (-e^{\I z} + e^{-\I z})/2 \\
            1/\I &=& -\I \\
            \cos_x(z) &=& \cos_y(z)/\I
        \end{eqnarray*}
        \end{proof}
        \item %3.40
        \begin{eqnarray*}
            \sin(z) &=& (e^{\I z}-e^{-\I z})/(2 \I) \\
            \overline{\sin{z}} &=& \overline{(e^{\I z}-e^{-\I z})/(2 \I)} \\
                               &=& \overline{(e^{\I(x+y\I)}-e^{-\I(x+y\I)})/(2 \I)} = \overline{(e^{-y+x\I}-e^{y-x\I})/(2 \I)} \\
                               &=& \overline{(e^{-y}(\cos(x)+\I\sin(x))-e^{y}(\cos(-x)+\I\sin(-x)))/(2\I)} \\
                               &=& \overline{(e^{-y}(-\I\cos(x)+\sin(x))-e^{y}(-\I\cos(-x)+\sin(-x)))/(2)} \\
                               &=& (e^{-y}(\I\cos(x)+\sin(x))-e^y(\I\cos(x)-\sin(x)))/2 \\
            \sin(\bar{z}) &=& (e^{\I \bar{z}}-e^{-\I \bar{z}})/(2 \I) =  (e^{\I (x-y\I)}-e^{-\I (x-y\I)})/(2 \I) \\
                          &=& (e^{y+x\I}-e^{-y-x\I})/(2 \I) \\
                          &=& (e^y(\cos(x)+\I\sin(x))-e^{-y}(\cos(-x)+\I\sin(-x)))/(2\I) \\
                          &=& (e^y(-\I\cos(x)+\sin(x))-e^{-y}(-\I\cos(-x)+\sin(-x)))/2  \\
                          &=& (-e^y(\I\cos(x)-\sin(x))+e^{-y}(\I\cos(x)+\sin(x)))/2 \\
            \overline{\sin{z}} &=& \sin(\bar{z})
        \end{eqnarray*}
        \setcounter{enumii}{50}
        \item %3.51
        \begin{eqnarray*}
            1 &=& \frac{d z}{dz} \\
              &=& \frac{d e^{\log{z}}}{dz} \\
              &=& e^{\log{z}}\frac{d\log{z}}{dz} \\
              &=& z \frac{d\log{z}}{dz} \\
         \frac{1}{z} &=& \frac{d\log{z}}{dz}
        \end{eqnarray*}
        \setcounter{enumii}{54}
        \item %3.55
        \begin{eqnarray*}
            \frac{d z^w}{dz} &=& \frac{d e^{w\log{z}}}{dz} \\
                             &=& e^{w\log{z}}\frac{d w\log{z}}{dz} \\
                             &=& z^w w/z \\
                             &=& wz^{w-1}
        \end{eqnarray*}
        \setcounter{enumii}{56}
        \item %3.57
        \begin{enumerate}[label=(\alph*)]
            \item %3.57.a
                $0$
            \item %3.57.b
                $-\ln(2)-\I\pi/2$
            \item %3.57.c
                $-\ln(2)+\I\pi$
            \item %3.57.d
                $-\pi/2$
            \item %3.57.e
                $\ln(2)/2+3\I\pi/4$
            \item %3.57.f
                $ 1.60944+0.927295\I$
        \end{enumerate}
        \setcounter{enumii}{58}
        \item %3.59
        \begin{enumerate}[label=(\alph*)]
            \item %3.59.a
                $\cos(\pi \sqrt{2}) + \I \sin(\pi \sqrt{2})$
            \item %3.59.b
                $2\cos(\ln(2))-2\I\sin(\ln(2))$
            \item %3.59.c
                $2^{1/\sqrt{2}} \cos(\frac{\pi\sqrt2}{4}) + 2^{1/\sqrt{2}} \I\sin(\frac{\pi\sqrt2}{4})$
            \item %3.59.d
                $e^{-\pi/2}$
            \item %3.59.e
                $\pi \I$
            \item %3.59.f
                $e^{\pi/2}$
        \end{enumerate}
    \end{enumerate}
\end{enumerate}
\end{document}

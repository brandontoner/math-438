\documentclass{article}%
\usepackage{amsmath}%
\usepackage{amsfonts}%
\usepackage{amssymb}%
\usepackage{graphicx}
\usepackage{enumitem}
%-------------------------------------------
\newtheorem{theorem}{Theorem}
\newtheorem{acknowledgement}[theorem]{Acknowledgement}
\newtheorem{algorithm}[theorem]{Algorithm}
\newtheorem{axiom}[theorem]{Axiom}
\newtheorem{case}[theorem]{Case}
\newtheorem{claim}[theorem]{Claim}
\newtheorem{conclusion}[theorem]{Conclusion}
\newtheorem{condition}[theorem]{Condition}
\newtheorem{conjecture}[theorem]{Conjecture}
\newtheorem{corollary}[theorem]{Corollary}
\newtheorem{criterion}[theorem]{Criterion}
\newtheorem{definition}[theorem]{Definition}
\newtheorem{example}[theorem]{Example}
\newtheorem{exercise}[theorem]{Exercise}
\newtheorem{lemma}[theorem]{Lemma}
\newtheorem{notation}[theorem]{Notation}
\newtheorem{problem}[theorem]{Problem}
\newtheorem{proposition}[theorem]{Proposition}
\newtheorem{remark}[theorem]{Remark}
\newtheorem{solution}[theorem]{Solution}
\newtheorem{summary}[theorem]{Summary}
\newcommand\abs[1]{\left|#1\right|}
\newenvironment{proof}[1][]{\begin{samepage}\textbf{Proof #1} }{\rule{0.5em}{0.5em} \end{samepage}}
\setlength{\textwidth}{7.0in}
\setlength{\oddsidemargin}{-0.35in}
\setlength{\topmargin}{-0.5in}
\setlength{\textheight}{9.0in}
\setlength{\parindent}{0.3in}
\begin{document}

\begin{flushright}
\textbf{Brandon Toner \\
\today}
\end{flushright}

\begin{center}
\textbf{MATH 438: Introduction to Complex Variables \\
Proofs} \\
\end{center}

\begin{enumerate}
    \item Chapter 1
    \begin{enumerate}[label*=\arabic*.]
        \item %1.1
        \begin{enumerate}[label=\alph*.]
            \item 
            \begin{proof}[$(a+bi)(a-bi)=a^2 + b^2$]
                \begin{eqnarray*}
                    (a+bi)(a-bi)&=& a^2 - abi + abi - b^2i^2 \\
                                &=& a^2 - b^2 i^2            \\
                                &=& a^2 + b^2
                \end{eqnarray*}
            \end{proof}
            \item
            \begin{proof}[$(a+bi)^2=a^2 - b^2 + 2abi$]
                \begin{eqnarray*}
                    (a+bi)^2 &=& (a+bi)(a+bi)              \\
                             &=& a^2 + abi + abi + b^2 i^2 \\
                             &=& a^2 - b^2 + 2abi
                \end{eqnarray*}
            \end{proof}
        \end{enumerate}
        \item
    	\begin{enumerate}[label=\alph*]
            \item
            \begin{proof}[$zw=wz$]
                \\ Let $z=x+yi$ and $w=u+vi$
                \begin{eqnarray*}
                    zw&=&(x+yi)(u+vi) \\
                      &=&ux+xvi+yui+yvi^2 \\
                      &=&ux-vy+(vx+uy)i \\
                    wz&=&(u+vi)(x+yi) \\
                      &=&ux+uyi+vxi+vyi^2 \\
                      &=&ux-vy+(vx+uy)i \\
                    zw&=&wz
                \end{eqnarray*}
            \end{proof}
            \item
            \begin{proof}[$z(r+w)=zr+zw$]
                \\ Let $z=x+yi$, $w=u+vi$, and $r=p+qi$
                \begin{eqnarray*}
                    z(r+w)&=&(x+yi)(p+qi+u+vi) \\
                          &=&px+qxi+ux+vxi+pyi+i^2yq+uyi+yvi^2 \\
                          &=&px-yq-vy+(qx+vx+py+uy)i \\
                    zr+zw &=&(x+yi)(p+qi)+(x+yi)(u+vi) \\
                          &=&px+qxi+pyi+i^2yq+ux+vxi+uyi+vyi^2 \\
                          &=&px-yq-vy+(qx+py+vx+uy)i \\
                    z(r+w)&=&zr+zw
                \end{eqnarray*}
            \end{proof}
            \item
            \begin{proof}[$r(wz)=(rw)z$]
                \\ Let $z=x+yi$, $w=u+vi$, and $r=p+qi$
                \begin{eqnarray*}
                    r(wz)&=&(p+qi)((u+vi)(x+yi)) \\
                         &=&(p+qi)(ux-vy+(vx+uy)i) \\
                         &=&pux-pvy+p(vx+uy)i+quxi-qvyi-q(vx+uy) \\
                         &=&pux-pvy-qvx-quy+(pvx+puy+qux-qvy)i \\
                    (rw)z&=&((p+qi)(u+vi))(x+yi) \\
                         &=&(pu-qv+(pv+qu)i)(x+yi) \\
                         &=&pux+puyi-qvx-qvyi+x(pv+qu)i-y(pv+qu) \\
                         &=&pux-pvy-qvx-quy+(puy-qvy+pvx+qux)i \\
                    r(wz)&=&(rw)z 
               \end{eqnarray*}
            \end{proof}
        \end{enumerate}
        \item
        \begin{enumerate}[label=\alph*]
            \item
            \begin{proof}[$\overline{z+w}=\overline{z}+\overline{w}$]
                \\ Let $z=x+yi$ and $w=u+vi$
                \begin{eqnarray*}
                    \overline{z+w}&=&\overline{x+u+(y+v)i} \\
                                  &=&x+u-(y+v)i \\
                    \overline{z}+\overline{w}&=&\overline{x+yi}+\overline{u+vi} \\
                                             &=&x-yi+u-vi \\
                                             &=&x+u-(y+v)i \\
                    \overline{z+w}&=&\overline{z}+\overline{w}
                \end{eqnarray*}
            \end{proof}
            \item
            \begin{proof}[$\overline{zw}=\bar{z}\bar{w}$]
                \\ Let $z=x+yi$ and $w=u+vi$
                \begin{eqnarray*}
                    \overline{zw}&=&\overline{(x+yi)(u+vi)} \\
                                 &=&\overline{xu-vy+(xv+uy)i} \\
                                 &=&xu-vy-(xv+uy)i \\
                    \bar{z}\bar{w}&=&(\overline{x+yi})(\overline{u+vi}) \\
                                  &=&(x-yi)(u-vi) \\
                                  &=&xu-vy-(vx+uy)i \\
                    \overline{zw}&=&\bar{z}\bar{w}
                \end{eqnarray*}
            \end{proof}
        \end{enumerate}
        \item
        \begin{enumerate}[label=\alph*]
            \item
            \begin{proof}[$\abs{z}^2=z\overline{z}$]
                \\ Let $z=x+yi$
                \begin{eqnarray*}
                    \abs{z}^2&=&\abs{x+yi}^2 \\
                             &=&x^2+y^2 \\
                    z\overline{z}&=&(x+yi)\overline{(x+yi)} \\
                                 &=&(x+yi)(x-yi) \\
                                 &=&x^2-xyi+xyi+y^2 \\
                                 &=&x^2+y^2 \\
                    \abs{z}^2&=&z\overline{z}
                \end{eqnarray*}
            \end{proof}
            \item
            \begin{proof}[$\abs{zw}=\abs{z}\abs{w}$]
                \\ Let $z=x+yi$ and $w=u+vi$
                \begin{eqnarray*}
                    \abs{zw}&=&\abs{(x+yi)(u+vi)} \\
                            &=&\abs{xu-vy+(xv+uy)i} \\
                            &=&\sqrt{(xu-vy)^2+(xv+uy)^2} \\
                            &=&\sqrt{u^2x^2-2uvxy+v^2y^2+u^2y^2+2uvxy+v^2x^2} \\
                            &=&\sqrt{u^2x^2+v^2y^2+u^2y^2+v^2x^2} \\
                    \abs{z}\abs{w}&=&\abs{x+yi}\abs{u+vi} \\
                                  &=&\sqrt{x^2+y^2}\sqrt{u^2+v^2} \\
                                  &=&\sqrt{(x^2+y^2)(u^2+v^2)} \\
                                  &=&\sqrt{x^2u^2+x^2v^2+y^2u^2+y^2v^2} \\
                    \abs{zw}&=&\abs{z}\abs{w}
                \end{eqnarray*}
            \end{proof}
            \item
            \begin{proof}[$\abs{z}=\abs{\overline{z}}$]
                \\ Let $z=x+yi$
                \begin{eqnarray*}
                    \abs{z}&=&\abs{x+yi} \\
                           &=&\sqrt{x^2+y^2} \\
                    \abs{\overline{z}}&=&\abs{\overline{x+yi}} \\
                                      &=&\abs{x-yi} \\
                                      &=&\sqrt{x^2+y^2} \\
                    \abs{z}&=&\abs{\overline{z}} 
                \end{eqnarray*}
            \end{proof}
        \end{enumerate}
        \item %1.5
        \begin{proof}[$z^{-1}=\frac{\overline{z}}{\abs{z}^2} \Leftrightarrow z z^{-1}=1$]
            \\ Let $z\neq0$
            \begin{eqnarray*}
                z^{-1}&=&\frac{\overline{z}}{\abs{z}^2} \\
                zz^{-1}&=&\frac{z\overline{z}}{\abs{z}^2} \\
                       &=&\frac{\abs{z}^2}{\abs{z}^2} \text{ by 1.4} \\
                       &=&1
            \end{eqnarray*}
        \end{proof}
        \item %1.6
        \begin{enumerate}[label=\alph*]
            \item %1.6.a
            \begin{proof}[$\overline{\frac{z}{w}}=\frac{\bar{z}}{\bar{w}}$]
                \\ Let $z=x+yi$ and $w=u+vi$
                \begin{eqnarray*}
                    \overline{(\frac{z}{w})}&=&\overline{(\frac{x+yi}{u+vi})} \\
                                            &=&\overline{(\frac{x+yi}{u+vi}\frac{u-vi}{u-vi})} \\
                                            &=&\overline{(\frac{xu+yv+(uy-xv)i}{u^2+v^2})} \\
                                            &=&\frac{xu+yv-(uy-xv)i}{u^2+v^2} \\
                    \frac{\bar{z}}{\bar{w}}&=&\frac{\overline{x+yi}}{\overline{u+vi}} \\
                                           &=&\frac{x-yi}{u-vi} \\
                                           &=&\frac{(x-yi)(u+vi)}{(u-vi)(u+vi)} \\
                                           &=&\frac{xu+yv-(uy-xv)i}{u^2+v^2} \\ 
                    \overline{(\frac{z}{w})}&=&\frac{\bar{z}}{\bar{w}}
                \end{eqnarray*}
            \end{proof}
            \item %1.6.b
            \begin{proof}[$\abs{\frac{z}{w}}=\frac{\abs{z}}{\abs{w}}$]
                \\ Let $z=x+yi$ and $w=u+vi$
                \begin{eqnarray*}
                    \abs{\frac{z}{w}}&=&\abs{\frac{x+yi}{u+vi}} \\
                                     &=&\abs{\frac{(x+yi)(u-vi)}{(u+vi)(u-vi)}} \\
                                     &=&\abs{\frac{xu+vy+(uy-xv)i}{u^2+v^2}} \\
                                     &=&\abs{\frac{xu+vy}{u^2+v^2}+\frac{(uy-xv)i}{u^2+v^2}} \\
                                     &=&\sqrt{(\frac{xu+vy}{u^2+v^2})^2 + (\frac{uy-xv}{u^2+v^2})^2 } \\
                                     &=&\sqrt{\frac{(xu+vy)^2+(uy-vx)^2}{(u^2+v^2)^2}} \\
                                     &=&\sqrt{\frac{(x^2+y^2)(u^2+v^2)}{(u^2+v^2)^2}} \\
                                     &=&\sqrt{\frac{x^2+y^2}{u^2+v^2}} \\
                                     &=&\frac{\sqrt{x^2+y^2}}{\sqrt{u^2+v^2}} \\
                                     &=&\frac{\abs{z}}{\abs{w}}
                \end{eqnarray*}
            \end{proof}
        \end{enumerate}
        \item %1.7
        \begin{enumerate}[label=\alph*]
            \item %1.7.a
            \begin{proof}[$\abs{\frac{z-1}{z+1}}^2=\frac{(x-1)^2+y^2}{(x+1)^2+y^2}$]
                \\ Let $z=x+yi$
                \begin{eqnarray*}
                    \abs{\frac{z-1}{z+1}}^2&=&\abs{\frac{x-1+yi}{x+1+yi}}^2 \\
                                           &=&\frac{\abs{x-1+yi}}{\abs{x+1+yi}} \text{ by 1.6.b} \\
                                           &=&\frac{(x-1)^2+y^2}{(x+1)^2+y^2}
                \end{eqnarray*}
            \end{proof}
            \item %1.7.b
            \begin{proof}[$\abs{\frac{1+4i}{4+i}}=1$]
                \begin{eqnarray*}
                    \abs{\frac{1+4i}{4+i}}&=&\frac{\abs{1+4i}}{\abs{4+i}} \text{ by 1.6.b} \\
                                          &=&\frac{\sqrt{1^2+4^2}}{\sqrt{4^2+1^2}} \\
                                          &=&1
                \end{eqnarray*}
            \end{proof}
            \item %1.7.c
            \begin{proof}[$\abs{\cos(\theta)+i \sin(\theta)}=1$]
                \begin{eqnarray*}
                    \abs{\cos(\theta)+i \sin(\theta)}&=&\sqrt{\cos^2(\theta)+\sin^2(\theta)} \\
                                                     &=&1
                \end{eqnarray*}
            \end{proof}
            \item %1.7.d
            \begin{proof}[$\abs{\frac{3+4i}{4+3i}}=1$]
                \begin{eqnarray*}
                    \abs{\frac{3+4i}{4+3i}}&=&\frac{\abs{3+4i}}{\abs{4+3i}} \text{ by 1.6.b} \\
                                           &=&\frac{\sqrt{3^2+4^2}}{\sqrt{4^2+3^2}} \\
                                           &=&1
                \end{eqnarray*}
            \end{proof}
        \end{enumerate}
        \item %1.8
        \begin{enumerate}[label=\alph*]
            \item %1.8.a
            \begin{proof}[$\frac{1}{2+i}=\frac{2}{5}-\frac{1}{5}i$]
                \begin{eqnarray*}
                    \frac{1}{2+i}&=&\frac{1}{2+i}\frac{2-i}{2-i} \\
                                 &=&\frac{2-i}{2^2+1^2} \\
                                 &=&\frac{2}{5}-\frac{1}{5}i
                \end{eqnarray*}
            \end{proof}
            \item %1.8.b
            \begin{proof}[$\frac{(1+i)^2}{3+2i}=\frac{4}{13}+\frac{6}{13}i$]
                \begin{eqnarray*}
                    \frac{(1+i)^2}{3+2i}&=&\frac{2i}{3+2i} \\
                                        &=&\frac{2i}{3+2i} \frac{3-2i}{3-2i} \\
                                        &=&\frac{4+6i}{3^2+2^2} \\
                                        &=&\frac{4}{13}+\frac{6}{13}i
                \end{eqnarray*}
            \end{proof}
            \item %1.8.c
            \begin{proof}[$\frac{2+i}{3+4i}=\frac{2}{5}-\frac{1}{5}i$]
                \begin{eqnarray*}
                    \frac{2+i}{3+4i}&=&\frac{2+i}{3+4i} \frac{3-4i}{3-4i} \\
                                    &=&\frac{10-5i}{3^2+4^2} \\
                                    &=&\frac{2}{5}-\frac{1}{5}i
                \end{eqnarray*}
            \end{proof}
            \item %1.8.d
            \begin{proof}[$\frac{1}{z^2}=\frac{\overline{z^2}}{(z\overline{z})^2}$]
                \\ Let $z=x+yi$
                \begin{eqnarray*}
                    \frac{1}{z^2}&=&\frac{1}{(x+yi)^2} \\
                                 &=&\frac{1}{x^2-y^2+2xyi} \\
                                 &=&\frac{1}{x^2-y^2+2xyi}\frac{x^2-y^2-2xyi}{x^2-y^2-2xyi} \\
                                 &=&\frac{x^2-y^2-2xyi}{(x^2-y^2)^2} \\
                                 &=&\frac{\overline{z^2}}{(z\overline{z})^2}
                \end{eqnarray*}
            \end{proof}
        \end{enumerate}
        \item %1.9
        \begin{proof}[$2c=z\xi+\overline{z\xi} \Leftrightarrow c=ax-by$]
            \\ Let $z=x+yi$ and $\xi=a+bi$
            \begin{eqnarray*}
                2c&=&z\xi + \overline{z\xi} \\
                  &=&(x+yi)(a+bi) + \overline{(x+yi)(a+bi)} \\
                  &=&ax-by+(ay+bx)i+\overline{ax-by+(ay+bx)i} \\
                  &=&ax-by+(ay+bx)i+ax-by-(ay+bx)i \\
                  &=&2(ax-by) \\
                 c&=&ax-by
            \end{eqnarray*}
            $c$ must be real, since $a, b, x$, and $y$ are real. The slope is $\frac{Re(\xi)}{Im(\xi)}$
        \end{proof}
        \item %1.10
        \begin{proof}[$z\bar{z}-\xi\bar{z}-\bar{\xi}z=c$ is the equation of a circle centered at $\xi$ with radius $\sqrt{c + \abs{\xi}^2}$]
            \begin{eqnarray*}
                z\bar{z}-\xi\bar{z}-\bar{\xi}z                 &=& c \\
                z\bar{z}-\xi\bar{z}-\bar{\xi}z + \xi \bar{\xi} &=& c + \xi \bar{\xi} \\
               (z - \xi)\overline(x-\xi) &=& c + \abs{\xi}^2 \\
               \abs{z - \xi}^2 &=& c + \abs{\xi}^2 \\
               \abs{z - \xi} &=& \sqrt{c + \abs{\xi}^2}
            \end{eqnarray*}
        \end{proof}
        \item %1.11
        \begin{enumerate}[label=\alph*]
            \item $\abs{z - 1 - i} = 2$
            \item The statements $x=5$ and $ax-by=c$ are equivalent when $a=1$, $b=0$, and $c=5$. Therefore, by 1.9, the complex equation is $z+\overline{z}=10$.
            \item $y=-2 \Leftrightarrow ax-by=c$ when $a=0, b=-1$, and $c=-2$. Therefore, by 1.9, the complex equation is $-iz+\overline{-iz}=-4$
            \item $y-2x=0 \Leftrightarrow ax-by=c$ when $a=-2, b=1$, and $c=0$. Therefore, by 1.9, the complex equation is $(-2+i)z+\overline{(-2+i)z}=0$.
        \end{enumerate}
        \item %1.12
        \begin{enumerate}[label=\alph*]
            \item %1.12.a
                $z+\overline{z}=2 \Leftrightarrow 2x=2 \Leftrightarrow x=1$
        \end{enumerate}
        \item %1.13
        \begin{enumerate}[label=\alph*]
            \item %1.13.a
                $Arg(1) = 0$. $arg(1)=2\pi k$
            \item %1.13.b
                $Arg(-3)=\pi$. $arg(-3)=\pi + 2\pi k$
            \item %1.13.c
                $Arg(-1+i)=\frac{3\pi}{4}$. $arg(-1+i)=\frac{3\pi}{4}+2\pi k$
            \item %1.13.d
                $Arg(3+3i)=\frac{\pi}{4}$. $arg(3+3i)=\frac{\pi}{4}+2\pi k$
            \item %1.13.e
                $Aarg(1-\sqrt{3}i)=-\frac{\pi}{3}$. $arg(1-\sqrt{3}i)=-\frac{\pi}{3}+2\pi k$
            \item %1.13.f
                $Arg(-4i)=-\frac{\pi}{2}$. $arg(-4i)=-\frac{\pi}{2}+2\pi k$
        \end{enumerate}
        \item %1.14
        \begin{enumerate}[label=\alph*]
            \item %1.14.a
                $1=1$
            \item %1.14.b
                $-3=3e^{i\pi}$
            \item %1.14.c
                $-1+i=\sqrt{2}e^{i\frac{3\pi}{4}}$
            \item %1.14.d
                $3+3i=3\sqrt{2}e^{i\frac{\pi}{4}}$
            \item %1.14.e
                $1-\sqrt{3}i=2e^{-i\frac{\pi}{3}}$
            \item %1.14.f
                $-4i=4e^{-i\frac{\pi}{2}}$
        \end{enumerate}
        \item %1.15
        \begin{proof}[$z \neq 0, \theta \in arg(z) \implies Re z=\abs{z}\cos(\theta) and Im z = \abs{z}\sin(\theta)$]
            Let $\theta \in arg(z)$.
            The polar form for $z$ can then be written as.
            \begin{eqnarray*}
                \abs{z} e^{i \theta} &=& \abs{z} (\cos(\theta) + i \sin(\theta)) \\
                                     &=& \abs{z} \cos(\theta) + i\abs{z}\sin(\theta) \\
                                Re z &=& \abs{z} \cos(\theta) \\
                                Im z &=& \abs{z} \sin(\theta)
            \end{eqnarray*}
        \end{proof}
        \item %1.16
            $\overline{z}$ is the reflection of $z$ over the real axis.  Therefore, $Arg(\overline{z})=-Arg(z)$ if $z$ isn't on the negative real axis, including zero. Same with $arg(\overline{z})=-arg(z)$.
        \setcounter{enumii}{22}
        \item %1.23
        \begin{enumerate}[label=\alph*]
            \item %1.23.a
                $(1+i)^5=\sqrt{2}^5(\cos(\frac{5 \pi}{4}) + i \sin(\frac{5 \pi}{4}))$
            \item %1.23.b
                $(1+\sqrt{3}i)^5=32(\cos(\frac{5 \pi}{3}) + i \sin(\frac{5 \pi}{3}))$
            \item %1.23.c
                $(1+i)^{24}=4096(\cos(\frac{24 \pi}{4}) + i \sin(\frac{24 \pi}{4}))$
        \end{enumerate}
        \item %1.24
        \item %1.25
        \item %1.26
        \begin{enumerate}[label=\alph*]
            \item %1.26.a
            \begin{proof}[$\abs{e^{i \theta}}=1$]
                \begin{eqnarray*}
                    \abs{e^{i \theta}} &=& \abs{\cos(\theta) + i \sin(\theta)} \\
                                       &=& \sqrt{\cos^2(\theta) + \sin^2(\theta)} \\
                                       &=& 1
                \end{eqnarray*}
            \end{proof}
            \item %1.26.b
            \begin{proof}[$\overline{e^{i\theta}}=e^{-i\theta}$]
                \begin{eqnarray*}
                    \overline{e^{i\theta}}&=& \overline{\cos(\theta) + i \sin(\theta)} \\
                                          &=& \cos(\theta) - i \sin(\theta) \\
                                          &=& \cos(\theta) + i \sin(-\theta) \\
                                          &=& \cos(-\theta) + i \sin(-\theta) \\
                                          &=& e^{-i\theta}
                \end{eqnarray*}
            \end{proof}
        \end{enumerate}
        \item %1.27
        \begin{enumerate}[label=\alph*]
            \item %1.27.a
            \begin{proof}[$e^{i\theta}e^{i\phi}=e^{i(\theta+\phi)}$]
                \begin{eqnarray*}
                    e^{i\theta}e^{i\phi} &=& (\cos(\theta) + i\sin(\theta))(\cos(\phi) + i\sin(\phi)) \\
                                         &=& \cos(\theta + \phi) + i\sin(\theta + \phi) \\
                                         &=& e^{i(\theta+\phi)}
                \end{eqnarray*}
            \end{proof}
        \end{enumerate}
        \item %1.28
        \begin{enumerate}[label=\alph*]
            \item %1.28.a
            \begin{proof}[$e^ze^w=e^{z+w}$]
                \begin{eqnarray*}
                    e^z e^w &=& e^{x+yi} e^{u+vi} \\
                            &=& e^x e^{yi} e^u e^{vi} \\ 
                            &=& e^{x + u} e^{vi + yi} \\
                            &=& e^{x + u + vi + yi} \\
                            &=& e^{z+w}
                \end{eqnarray*}
            \end{proof}
        \end{enumerate}
        \setcounter{enumii}{32}
        \item %1.33
        \begin{proof}
            \begin{align*}
                1 &= z^n \\ 
                  &= (\abs{z} (\cos(arg(z)) + i \sin(arg(z))))^n \\
                  &= \abs{z}^n (\cos(n arg(z)) + i \sin(n arg(z))) \\
                  &= \cos(n arg(z)) + i \sin(n arg(z)) & \abs{z} \text{ must be 1} \\
                  &= \cos(n arg(z))\\
          2 \pi k &= n arg(z) \\
          (2\pi k)/n &= arg(z) \\
                z &= e^{i arg(z)} = e^{\frac{2 \pi i k}{n}}
            \end{align*}
        \end{proof}
        \item %1.34
            $1, i, -1, -i$
        \item %1.35
            $1, (1+i\sqrt{3})/2, (-1+i\sqrt{3})/2, -1,  (-1-i\sqrt{3})/2, (1-i\sqrt{3})/2$
    \end{enumerate}
\end{enumerate}
\end{document}

\documentclass{article}%
\usepackage{amsmath}%
\usepackage{amsfonts}%
\usepackage{amssymb}%
\usepackage{graphicx}
\usepackage{enumitem}
%-------------------------------------------
\newtheorem{theorem}{Theorem}
\newtheorem{acknowledgement}[theorem]{Acknowledgement}
\newtheorem{algorithm}[theorem]{Algorithm}
\newtheorem{axiom}[theorem]{Axiom}
\newtheorem{case}[theorem]{Case}
\newtheorem{claim}[theorem]{Claim}
\newtheorem{conclusion}[theorem]{Conclusion}
\newtheorem{condition}[theorem]{Condition}
\newtheorem{conjecture}[theorem]{Conjecture}
\newtheorem{corollary}[theorem]{Corollary}
\newtheorem{criterion}[theorem]{Criterion}
\newtheorem{definition}[theorem]{Definition}
\newtheorem{example}[theorem]{Example}
\newtheorem{exercise}[theorem]{Exercise}
\newtheorem{lemma}[theorem]{Lemma}
\newtheorem{notation}[theorem]{Notation}
\newtheorem{problem}[theorem]{Problem}
\newtheorem{proposition}[theorem]{Proposition}
\newtheorem{remark}[theorem]{Remark}
\newtheorem{solution}[theorem]{Solution}
\newtheorem{summary}[theorem]{Summary}
\newcommand\abs[1]{\left|#1\right|}
\newcommand\I{\textbf{i}}
\newenvironment{proof}[1][]{\begin{samepage}\textbf{Proof #1} }{\rule{0.5em}{0.5em} \end{samepage}}
\setlength{\textwidth}{7.0in}
\setlength{\oddsidemargin}{-0.35in}
\setlength{\topmargin}{-0.5in}
\setlength{\textheight}{9.0in}
\setlength{\parindent}{0.3in}
\begin{document}

\begin{flushright}
\textbf{Brandon Toner \\
\today}
\end{flushright}

\begin{center}
\textbf{MATH 438: Introduction to Complex Variables \\
Proofs} \\
\end{center}

\begin{enumerate}
    \setcounter{enumi}{1}
    \item Chapter 2
    \begin{enumerate}[label*=\arabic*.]
        \item %2.1
        \begin{enumerate}[label=\alph*.]
            \item %2.1.a
                Real dot: $8$ \\
                Complex dot: $8+\I$
            \item %2.1.b
                Real dot: $8$ \\
                Complex dot: $8-\I$
             \item %2.1.c
                Real dot: $1$ \\
                Complex dot: $i+2\I$
            \item %2.1.d
                Real dot: $2$ \\
                Complex dot: $2-\I$
            \item %2.1.e
                Real dot: $x$ \\
                Complex dot: $x+y\I$
            \item %2.1.f
                Real dot: $y$ \\
                Complex dot: $y-x\I$
        \end{enumerate}
        \item %2.2
        \begin{eqnarray*}
            Re(z\bar{w})&=&Re(xu+yv+\I(yu-vx))=xu+yv \\
            Re(\bar{z}w)&=&Re(xu+yv+\I(vx-uy))=xu+yv
        \end{eqnarray*}
        \item %2.3
        \begin{proof}[$\abs{z \cdot w} \leq \abs{z}\abs{w}$]
            \begin{eqnarray*}
                \abs{Re(z)} &\leq& \abs{z} \\
                \abs{Im(z)} &\leq& \abs{z} \\
                \abs{z \cdot w} &=& \abs{Re(z\bar{w})} \\
                              &\leq& \abs{z \bar{w}} \\
                              &=& \abs{z}\abs{w}
            \end{eqnarray*}
        \end{proof}
        \item %2.4
        \begin{enumerate}[label=\alph*]
            \item % 2.4.a
            \begin{proof}[$\abs{z+w}^2 = \abs{z}^2 + \abs{w}^2 + 2 Re(z\bar{w})$]
                \begin{eqnarray*}
                    \abs{z+w}^2 &=& (z+w)\overline{(z+w)} \\
                                &=& z \bar{z} + w \bar{w} + z\bar{w} + \bar{z} w \\
                                &=& \abs{z}^2 + \abs{w}^2 + 2 Re(z\bar{w}) 
                \end{eqnarray*}
            \end{proof}
            \item % 2.4.b
            \begin{proof}[$\abs{z-w}^2 = \abs{z}^2 + \abs{w}^2 - 2 Re(z\bar{w})$]
                \begin{eqnarray*}
                    \abs{z-w}^2 &=& (z-w)\overline{(z-w)} \\
                                &=& z \bar{z} + w \bar{w} - z\bar{w} - \bar{z} w \\
                                &=& \abs{z}^2 + \abs{w}^2 - 2 Re(z\bar{w}) 
                \end{eqnarray*}
            \end{proof}
        \end{enumerate}
        \item %2.5
        \begin{proof}[$\abs{z+w}\leq\abs{z}+\abs{w}$]
            \begin{eqnarray*}
                \abs{z+w}^2 &=& \abs{z}^2 + \abs{w}^2 + 2 Re(z\bar{w}) \\
                            &\leq& \abs{z}^2 + \abs{w}^2 + \abs{z}\abs{w} \\
                            &=& (\abs{z}+\abs{w})^2 \\
                \implies \\
                \abs{z+w} &\leq& \abs{z}+\abs{w}
            \end{eqnarray*}
        \end{proof}
        \item %2.6
        \begin{proof}[$\abs{\abs{z}-\abs{w}}\leq\abs{z-w}$]
            \begin{eqnarray*}
                \abs{z} &=& \abs{z - w + w} \\
                        &\leq& \abs{z-w} +\abs{w} \\
                \abs{z}-\abs{w} &\leq& \abs{z-w}
            \end{eqnarray*}
        \end{proof}
        \item %2.7
            Proof by induction using triangle inequality.
        \item %2.8
        \item %2.9
        \begin{eqnarray*}
            p(z) &=& a_n z^n + a_{n-1} z^{n-1} + ... + a_1 z+ a_0 = 0 \\
            p(\bar{z}) &=& a_n \bar{z}^n + a_{n-1} \bar{z}^{n-1} + ... + a_1 \bar{z}+ a_0 \\ 
                       &=& \overline{P(z)} \\
                       &=& 0 
        \end{eqnarray*}
        \item %2.10
        \begin{proof} 
            \\
            Pick $\epsilon > 0$. Since $\abs{w_n} \to 0$, $\exists N$ such that $n\geq N \implies \abs{w_n}< \epsilon$. Therefore, $\abs{z_n - z} \leq \abs{w_n} < \epsilon$ \\
            $\implies z_n \to z$.\\
        \end{proof}
        \item %2.11
        Proof showing it $z_n$ can't get close enought to $z$ otherwise.
        \item %2.12
        For each $\epsilon, \exists N$ such that $\abs{z_n - z} < \epsilon$ for $n>N$. Thus $\abs{z_n} < \epsilon + \abs{z} \implies z_n$ is bounded for $n>N$.
        Since $z_n n\in1..N$ is finite, it must be bounded. Therefore $z_n$ is  bounded.
        \item %2.13
        \item %2.14
            $\frac{1-1}{1+1}=0$
        \item %2.15
            $\frac{z_n^2 - 1}{z_n - 1} = z_n+1 = 2$
        \item %2.16
            $1$
        \item %2.17
            $0$
        \item %2.18
            $z_n \to \infty$ if $\forall K>0\ \exists N$ such that $n>N \implies \abs{z_n}\geq K$
        \setcounter{enumii}{24}
        \item %2.25
            The sum of two continuous functions must be continuous, so $f(z)$ is continuous. \\
            Since $u$ and $\I v$ are on seperate axis, their addition can't be continuous if either one isn't continuous.
        \setcounter{enumii}{27}
        \item %2.28
        \begin{eqnarray*}
            z &=& \frac{z-w}{h} \\
              &=& \frac{z}{h}-\frac{w}{h} \\
           z\frac{h-1}{h} &=& -\frac{w}{h} \\
           z &=& -\frac{w}{h-1} \\
             &=& \frac{w}{1-h} 
        \end{eqnarray*}
    \end{enumerate}
\end{enumerate}
\end{document}

\documentclass{article}%
\usepackage{amsmath}%
\usepackage{amsfonts}%
\usepackage{amssymb}%
\usepackage{graphicx}
\usepackage{enumitem}
%-------------------------------------------
\newtheorem{theorem}{Theorem}
\newtheorem{acknowledgement}[theorem]{Acknowledgement}
\newtheorem{algorithm}[theorem]{Algorithm}
\newtheorem{axiom}[theorem]{Axiom}
\newtheorem{case}[theorem]{Case}
\newtheorem{claim}[theorem]{Claim}
\newtheorem{conclusion}[theorem]{Conclusion}
\newtheorem{condition}[theorem]{Condition}
\newtheorem{conjecture}[theorem]{Conjecture}
\newtheorem{corollary}[theorem]{Corollary}
\newtheorem{criterion}[theorem]{Criterion}
\newtheorem{definition}[theorem]{Definition}
\newtheorem{example}[theorem]{Example}
\newtheorem{exercise}[theorem]{Exercise}
\newtheorem{lemma}[theorem]{Lemma}
\newtheorem{notation}[theorem]{Notation}
\newtheorem{problem}[theorem]{Problem}
\newtheorem{proposition}[theorem]{Proposition}
\newtheorem{remark}[theorem]{Remark}
\newtheorem{solution}[theorem]{Solution}
\newtheorem{summary}[theorem]{Summary}
\newcommand\abs[1]{\left|#1\right|}
\newcommand\EXP[1]{\exp\left(#1\right)}
\newenvironment{proof}[1][]{\begin{samepage}\textbf{Proof #1} \\ }{\\ \rule{0.5em}{0.5em} \end{samepage} \\}
\setlength{\textwidth}{7.0in}
\setlength{\oddsidemargin}{-0.35in}
\setlength{\topmargin}{-0.5in}
\setlength{\textheight}{9.0in}
\setlength{\parindent}{0.3in}
\begin{document}

\begin{flushright}
\textbf{Brandon Toner \\
\today}
\end{flushright}

\begin{center}
\textbf{MATH 438: Introduction to Complex Variables \\
Assignment 1} \\
\end{center}

\begin{enumerate}
    \setcounter{enumi}{4}
    \item %5
    \begin{enumerate}[label=\alph*]
        \item %5.a
        \begin{proof}[$1 + z + z^2 + ... + z^n = \frac{1 - z^{n+1}}{1-z}$]
            \begin{align*}
                1 + z + z^2 + ... + z^n &= \sum\limits_{j=0}^n (z^j) \\
                                        &= \frac{1 - z}{1-z} \sum\limits_{j=0}^n (z^j) \\
                                        &= \frac{1}{1-z} \left( \sum\limits_{j=0}^n (z^j) - z \sum\limits_{j=0}^n (z^j) \right)  = \frac{1}{1-z} \left( \sum\limits_{j=0}^n (z^j) - \sum\limits_{j=0}^{n+1} (z^j) \right) \\
                                        &= \frac{1}{1-z} \left( z^0 - z^{n+1} \right) \\
                                        &= \frac{1 - z^{n+1}}{1-z}
            \end{align*}
        \end{proof}
    \end{enumerate}
    \item
    \begin{enumerate}[label=\alph*]
        \item %6.a
        \begin{proof}[$\prod\limits_{j=0}^{n-1}{(z-\omega_j)} = z^n - 1$]
            The roots of the function $z^n - 1$ are $\omega_0, \omega_1, ... ,\omega_{n-1}$.
            Therefore, the functions $\prod\limits_{j=0}^{n-1}{(z-\omega_j)}$ and $z^n - 1$ must be proportional.
            Since the coefficient of the $z^n$ term of $\prod\limits_{j=0}^{n-1}{(z-\omega_j)}$ must be $1$, the functions must be equal.
        \end{proof}
        \item %6.b
        \begin{proof}[$\omega_0+...+\omega_{n-1} = 0$]
        \begin{align*}
            \omega_j &= \EXP{\frac{2 \pi j i}{n}}\\
            \omega_0+...+\omega_{n-1} &= \EXP{0} + \EXP{\frac{2\pi i}{n}} +  \EXP{\frac{4\pi i}{n}} +  ... + \EXP{\frac{2 \pi (n-1) i}{n}} \\
                                      &= \EXP{\frac{2\pi i}{n}}^0+\EXP{\frac{2\pi i}{n}}^1+\EXP{\frac{2\pi i}{n}}^2+...+\EXP{\frac{2\pi i}{n}}^{n-1} \\
                                      &= \sum\limits_{j=0}^{n-1} \EXP{\frac{2\pi i}{n}}^j \\
                                      &= \frac{1-\EXP{\frac{2\pi i}{n}}^n}{1 - \EXP{\frac{2\pi i}{n}}} = \frac{1-\EXP{2\pi i}}{1 - \EXP{\frac{2\pi i}{n}}} \\
                                      &= 0
        \end{align*}
        \end{proof}
        \item %6.c
        \begin{proof}[$\prod\limits_{j=0}^{n-1} \omega_j = (-1)^{n-1}$]
            \begin{align*}
                \prod\limits_{j=0}^{n-1} \omega_j &= \prod\limits_{j=0}^{n-1} \EXP{\frac{2\pi ji}{n}} \\
                                                  &= \EXP{\sum\limits_{j=0}^{n-1}\frac{2\pi j i}{n}} = \EXP{\frac{2\pi i}{n} \sum\limits_{j=0}^{n-1}j} \\
                                                  &= \EXP{\frac{2\pi i}{n} \frac{n(n-1)}{2}} \\
                                                  &= \EXP{\pi (n - 1) i} \\
                                                  &= (-1)^{n-1}
            \end{align*}
        \end{proof}
        \item %6.d
        \begin{proof}[$\sum\limits_{j=0}^{n-1}{\omega_j^k}=\left\{ \begin{array} {lr} 0, & 1 \leq k \leq n-1 \\ n, & k=n \end{array} \right. $]
            \underline{Case 1}: $k=n$
            \begin{eqnarray*}
                \sum\limits_{j=0}^{n-1}{\omega_j^n} &=& \sum\limits_{j=0}^{n-1}{1} \\
                                                    &=& n
            \end{eqnarray*}
            \underline{Case 2}: $1 \leq k \leq n-1$
            \begin{align*}
                \sum\limits_{j=0}^{n-1}{\omega_j^k} &= \sum\limits_{j=0}^{n-1}{\EXP{\frac{2 \pi j i}{n}}^k} \\
                                                    &= \sum\limits_{j=0}^{n-1}{\EXP{\frac{2 \pi j k i}{n}}} \\
                                                    &= \sum\limits_{j=0}^{n-1}{\EXP{\frac{2 \pi k i}{n}}^j} \\
                                                    &= \frac{1-\EXP{\frac{2 \pi k i}{n}}^n}{1-\EXP{\frac{2\pi ki}{n}}}=\frac{1-\EXP{2\pi ki}}{1-\EXP{\frac{2 \pi k i}{n}}} \\
                                                    &= 0
            \end{align*}
        \end{proof}
    \end{enumerate}
\end{enumerate}
\end{document}
